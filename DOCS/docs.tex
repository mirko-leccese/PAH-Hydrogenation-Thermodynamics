%%-----------------------------------------------------------
%% 		DOCUMENT CLASS AND MARGIN DECLARATION		
\documentclass[a4paper, 11pt, twoside]{book}
\usepackage[top=4cm, bottom=4cm, left=3.5cm, right=3cm]{geometry}
%
%%-----------------------------------------------------------
%%-----------------------------------------------------------
%%		TEXT FORMATTING  & PAGE STYLE
\usepackage[english]{babel}
\usepackage[T1]{fontenc}
\usepackage[applemac]{inputenc}
\usepackage{enumerate}  
\usepackage{setspace}	%Setting the spaces between lines
\setstretch{1.0}
\usepackage{fancyhdr}
\pagestyle{fancy}
\fancyhead{}
\setlength{\headheight}{15pt}
\fancyhead[LE]{\nouppercase\leftmark}% LE -> Left part on Even pages
\fancyhead[RO]{\nouppercase\rightmark}% RO -> Right part on Odd pages
\fancyfoot[C]{\thepage}
\setlength{\parskip}{0pt}  %remove extra spaces between paragraphs
\newlength\tindent     %remove indent 
\setlength{\tindent}{\parindent}
\setlength{\parindent}{0pt}
\renewcommand{\indent}{\hspace*{\tindent}}
%%----------------------------------------------------------
%%----------------------------------------------------------
%%		TABLE & FIGURES
\usepackage{graphicx}
\usepackage{tabularx}
\usepackage{booktabs}
\usepackage{multirow}
\usepackage{float}
\usepackage[font=small,labelfont=bf,labelsep=period, tableposition=top,singlelinecheck=false]{caption}
%
%%-------------------------------------------------------
%%------------------------------------------------------
%%		MATH PACKAGES
\usepackage{amsmath}
\usepackage{amsxtra}
\usepackage{amstext}
\usepackage{amsthm}
\usepackage{amssymb}
\usepackage{amsfonts}
\usepackage{bm}
\usepackage{cancel}
\usepackage{siunitx}
%
%
\newcommand*{\bfrac}[2]{\genfrac{\lbrace}{\rbrace}{0pt}{}{#1}{#2}}
\DeclareMathOperator{\Imm}{Im}
\DeclareMathOperator{\Rea}{Re}
\DeclareMathOperator{\Tr}{Tr}
\DeclareMathOperator{\sign}{sgn}
%
%%---------------------------------------------
%%		BOX FOR EQUATIONS AND OTHER
\usepackage[leqno,fleqn,intlimits]{empheq}
\usepackage{fancybox}
\usepackage{color}
\definecolor{shadowcolor}{rgb}{0,.5,.5}
\setlength\shadowsize{2pt}
\usepackage[most]{tcolorbox}
\newtcolorbox[auto counter,number within=chapter]{mybox}[2][]{enhanced, breakable, width=\linewidth, sharp corners=all,  colback=white!95!black, ,fonttitle=\bfseries,
title=Box~\thetcbcounter: #2,#1}
%
%
\newtheorem{theorem}{Theorem}[section]
\newtheorem{corollary}{Corollary}[theorem]
\newtheorem{lemma}[theorem]{Lemma}
\newtheorem{Rule}{Rule}
%
%%-----------------------------------------------------
%%		NOTES
\usepackage[multiple]{footmisc}    %For multiple footnotes from the same source
\interfootnotelinepenalty=500000
%
%%Quote at the beginning of chapter
\makeatletter
\renewcommand{\@chapapp}{}% Not necessary...
\newenvironment{chapquote}[2][2em]
{\setlength{\@tempdima}{#1}%
\def\chapquote@author{#2}%
\parshape 1 \@tempdima \dimexpr\textwidth-2\@tempdima\relax\itshape}
 {\par\normalfont\hfill--\ \chapquote@author\hspace*{\@tempdima}\par\bigskip}
\makeatother
%%
%
%
%%----------------------------------------------------
%%       BIBLIOGRAPHY
%%\usepackage[super, square]{natbib}  %Super for superscripts and Square for square commas
%\usepackage{notoccite}
%%----------------------------------------------------------------------------------------
%%	LIST OF CONTENTS/FIGURES/TABLES PAGES
\setcounter{tocdepth}{2}  %1 to limit to section, 2 to subsection
\usepackage{color}   %May be necessary if you want to color links
\usepackage{hyperref}
\hypersetup{colorlinks=true,citecolor=black,linkcolor=black,linktocpage=true}
%%----------------------------------------------------------------------------------------

\begin{document}
\chapter*{Thermodynamics of PAH hydrogenation}

Let consider the following chemical equilibria:
\begin{gather}
    \text{C}_{20}\text{H}_{10+n} + H \rightarrow \text{C}_{20}\text{H}_{10+n+1} \nonumber \\
    2\text{H} \rightarrow \text{H}_2 \nonumber
\end{gather}
We define the \emph{fractional hydrogenation level} as
\begin{equation}
    f=\frac{N^0(H)}{N^0(H)/2+N^0(Co)} 
\end{equation}
where $N^0(H)$ is the total number of (extra) hydrogen available in a given sample and $N^0(Co)$ the total number of PAH molecules (bare \emph{plus} hydrogenated PAH molecules).
The free energy of $n$-fold hydrogenated structure is
\begin{equation}
    G_{n} = E_n - k_bT \ln{\mathcal{Z}^{int}_n}+k_bT\ln{\left(\frac{p_n}{\xi_n}\right)} \nonumber
\end{equation}
where $E_n$ is the DFT zero-point corrected energy, $\mathcal{Z}$ the internal partition function, $p_n$ the partial pressure and $\xi_n$ the thermal pressure, \emph{i.e.} $\xi_n=k_bT/\lambda_n^3$with $\lambda_n^3$ the \emph{De Broglie thermal wavelength}. The internal partition function accounting for rotational, vibrational and electronic contributions is given by
\begin{equation}
    \mathcal{Z}^{int}_{n} \simeq \frac{1}{\sigma_n}\left( \frac{T}{\theta}\right)^{3/2}\prod_{j=1}^{F_n}\left[1-\exp\left(-\frac{\hbar\omega_j^{(n)}}{k_bT}\right)\right]\mathcal{Z}_{n}^{el}  \label{eqapp:zint}
\end{equation}
where $\theta_n=\hbar^2/2I^{(n)}k_bT$ is the rotational temperature of the n-fold hydrogenated structure with $I^{(n)} = (\pi I^{(n)}_AI^{(n)}_BI^{(n)}_C)^{1/3}$ being its moment of inertia, given in terms of the principal values of inertia tensor, $\omega^{(n)}_j$ the normal mode frequencies ($j=1,2,...F_n$ where $F_n$ is the number of vibrational degrees of freedom, $F_n=3(36+n)-6$) and $\mathcal{Z}^{el}_n$ the electronic partition function, $\mathcal{Z}^{el}_n = 1+\mod(n,2)$.
For molecular hydrogen, the classical limit of rotational partition function is inappropriate at temperature $T \lesssim 100$ and one should consider the sum over \emph{all} states. To properly handle this situation, Equation \eqref{eqapp:zint} was still employed in the slightly modified form
\begin{equation}
    \mathcal{Z}^{int}_{H_2} \simeq \frac{1}{2}\frac{k_bT}{hcB}\left(1-\exp\left(-\frac{\hbar\omega}{k_bT}\right)\right)
\end{equation}
where the rotational constant $B$ was adjusted in order to make the entropy, $S=-k_bT\ln{\mathcal{Z}}$, continuous at 298.15 K (where $S$ was computed from the \emph{Shomate equation}\footnote{
Shomate Equation for standard entropy $S^{\circ}$ (J/mol K) reads as
\begin{equation*}
S^{\circ} = A \ln{(t)} + Bt + C\frac{t^2}{2} + D\frac{t^3}{3} - \frac{E}{2t^2} + G
\end{equation*}
where $t=T/1000$ is the reduced temperature (K) and $A,B,C,D,E,G$ are temperature-dependent parameters (their values can be found in the NIST Chemistry WebBook.}).

If one is interested in determining the most stable species at a given thermodynamic condition, one may consider the "formation reaction"
\begin{equation}
    \text{C}_{20}\text{H}_{10} + \frac{n}{2}\text{H}_{2} \rightarrow \text{C}_{20}\text{H}_{10+n} \nonumber
\end{equation}
and the corresponding Gibbs free energy 
\begin{align}
    \Delta G_f(P,T,f^0) = G_n&(P,T)-G_0(P-p_{hy},T)+ \nonumber \\
     &-\frac{n}{2}G_{H_2}(p_{hy},T) \nonumber 
\end{align}
where $p_{hy}=f^0P$ and look, for given conditions, $P$, $T$ and $f^0$, for the smallest values. 

\section*{Mixture free energy}
As a complementary approach to the thermodynamical analysis of PAH hydrogenation, let consider a mixture of PAH and (atomic/molecular) hydrogen and the problem of minimizing the mixture free energy. The initial mixture composition can be defined in terms of the number of molecules per each hydrogenated specie, $N^0_n$, plus the number of H atoms, $N^0_H$, and $\text{H}_2$ molecules, $N^0_{H_2}$. The associated initial Gibbs free energy is $G_{mix}^0$. The mixture composition can then be optimized \emph{stochastically}, by introducing random variations on each mixture components. After any variation,  the mixture free energy, $G_{mix}^{1}$ is re-computed and compared with that of the previous composition: if lower, $G_{mix}^1 < G_{mix}^0$, the newly random-generated composition is "saved" and used for the next step; otherwise that composition is "discarded" and another random-generated composition is checked until $G_{mix}^1 < G_{mix}^0$ is found. Such stochastic procedure can be applied to a mixture with an initial composition consisting of only bare PAH molecules and an excess of $\text{H}_2$, \emph{i.e.} setting $N^0_i\simeq 0$ and $N^0_{H}\simeq 0$. In principle, one may introduce random variations on each $N_i$ from the very first optimization step. However, a more reasonable choice consists in allowing the $(i+1)$ structure to "form" in the mixture only once the mixture with the hydrogenated molecules up to $n=i$ has reached the equilibrium\footnote{In other words, we are assuming that each hydrogenation step is "independent" of the others. Then, the equilibrium composition is reached through multiple "local" equilbria, $\text{C}_{20}\text{H}_{10+n} + \text{H} \rightarrow \text{C}_{20}\text{H}_{10+n+1}$.}. In the following, we describe in detail the computational strategy employed.

\subsection*{Computational strategy}
The composition of a mixture at a give step $i$ is specified by a set $C=[N^{(i)}_{\text{H}}, N^{(i)}_{\text{H}_2}, $ $N^{(i)}_0, .., N^{(i)}_M]$, where $N^{(i)}_{\text{H}}$ is  the number of H atoms, $N^{(i)}_{\text{H}_2}$ is the number of $\text{H}_2$ molecules and $N^{(i)}_n$ that of $n$-hydrogenated PAH molecules, where $n=0,1,..M$ with $M$ the total number of hydrogenation steps (\emph{i.e.} $M=24$ for coronene)\footnote{Remind that this is the mixture with the optimized composition generated by the $i-1$ step.}. Variations on the abundances are introduced by adding to these, $\delta N^{(i)}_H$, $\delta N^{(i)}_{\text{H}_2}$, $\delta N^{(i)}_n$, that are given by (\emph{e.g.} for $N^{(i)}_n$)
\begin{equation}
dN^{(i)}_n = \frac{1}{W}2\left( \xi - \frac{1}{2} \right)N^{(i)}_n \label{eqapp:variations}
\end{equation}
where $\xi \in [ 0,1]$ is a (pseudo)-random number and $W$ is a scaling factor that allows tuning the extent of random variations (for $W=1$, $dN^{(i)}_n\in [ -N^{(i)}_n, N^{(i)}_n]$). Henceforth, we omit the superscript $i$. To ensure the mass balance of PAH molecules, we first set
\begin{gather*}
\tilde{N}_n = N_n + dN_n \\
\tilde{N}_t = \sum_n^M \tilde{N}_n
\end{gather*}
and then we properly re-define the random-variated populations at step $i$ as 
\begin{equation}
N'_n = \tilde{N}_n \frac{N^0_t}{\tilde{N}_t}
\end{equation}
where $N^0_t$ is the total number of PAH molecules that has been fixed at the outset. Once the abundance of H atoms in the mixture is modified (according to Equation \eqref{eqapp:variations}), the new population of $\text{H}_2$ molecules is given by the mass conservation law
\begin{equation}
dN_{\text{H}_2}  = \frac{1}{2} \left( -dN_{\text{H}} - \sum_n^M ndN'_n \right)
\end{equation}
Notice that $dN'_n \neq dN_n$ are the final "true" random-variation, \emph{i.e.} $dN'_n = N'_n - N_n$. With the newly random-generated mixture composition, the mixture Gibbs free energy is computed according to
\begin{equation}
    G_{mix}^{(i)} = \sum_{n=0}^MN'_nG_n+N'_{H}G_{H}+N'_{H_2}G_{H_2} \nonumber
\end{equation}

%\begin{figure}
%\centering
%\includegraphics[width=0.68\textwidth]{APPENDIX/IMAGES/flowchart}
%\caption{Flowchart describing the optimization procedure.}
%\label{fig:flowchart}
%\end{figure}

As we mentioned before, we make the assumption of a stepwise optimization, where $[N_n, N_{n+1}, ..., N_{M}]$ populations are set to zero until $[N_{\text{H}}, N_{\text{H}_2}, N_0, ..., N_{n-1}]$ is optimized. In this sense, the full mixture composition optimization is divided into $M$ sub-optimizations, that we call $K=1, 2, ..., M$. The suboptimization $K=1$ involves the species $n=0,1$, $K=2$ the species $n=0,1,2$, etc. Since the interval of the random variation depends on the input composition at that step (see Equation \eqref{eqapp:variations}), at the very first step when the abundance of $N_n$ is "unconstrained", its value is set to a fraction (typically $1/100$ but this can be tuned in the input file) of the most abundant specie in the mixture at that moment. This choice guarantees that the equilibrium is hardly changed. Flowchart in Figure \ref{fig:flowchart} summarize the optimization procedure.

At the end of any sub-optimization, the reached equilibrium composition defined in terms of number of molecules ($N'_n$) is saved and fitted according to

\begin{equation}
    s(x)=\sum_{n=0}^{M} N'_n L_n(x) \nonumber 
\end{equation}
where $L_n(x)$ is a \emph{Lorentzian function}
\begin{equation}
    L_n(x) = \frac{1}{\pi}\frac{\frac{\gamma}{2}}{(x-n)^2+(\frac{\gamma}{2})^2} \nonumber
\end{equation}
with $\gamma$ is a scale parameter (the \emph{full width at half-maximum}) that allows tuning the shape of the lorentzian. 


\end{document}


